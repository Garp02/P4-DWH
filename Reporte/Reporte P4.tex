\documentclass[12pt]{article}

\setlength{\parskip}{5mm}

\usepackage[T1]{fontenc}
\usepackage[utf8]{inputenc}
\usepackage[spanish]{babel}

\usepackage{url}
\usepackage[top=2.5cm, bottom=2.5cm, left=1.7cm, right=1.7cm]{geometry}

\usepackage{amsmath, amsthm, amsfonts, amssymb, enumitem, float}
\usepackage{booktabs}
\usepackage[table]{xcolor}

\usepackage{graphicx}
\graphicspath{{./}}

\usepackage{fancyhdr}

\usepackage{tikz}
\usetikzlibrary{arrows,calc}
\tikzstyle{edge} = [shorten <= 2pt, shorten >= 2pt, >= stealth, line width = 1.1pt]
\tikzstyle{vertex} = [circle, draw, minimum size = 5pt, inner sep = 0pt,
outer sep = 0pt, line width = 0.7pt]

\usepackage{listings}

\pagestyle{fancy}
\setlength{\headheight}{15pt}
\fancyhf{}

\newcommand{\QED}{\begin{flushright}$\blacksquare$\end{flushright}}

% Colores para las tablas
\definecolor{headerblue}{HTML}{4F81BD}
\definecolor{headertext}{HTML}{FFFFFF}
\definecolor{rowgray}{HTML}{F2F2F2}

% Colores para los códigos
\definecolor{codegreen}{rgb}{0,0.6,0}
\definecolor{codegray}{rgb}{0.5,0.5,0.5}
\definecolor{codepurple}{rgb}{0.58,0,0.82}
\definecolor{backcolour}{rgb}{0.95,0.95,0.92}

\lstdefinestyle{mystyle}{
    backgroundcolor = \color{backcolour},
    commentstyle = \color{magenta},
    keywordstyle = \color{codegreen},
    numberstyle = \tiny\color{codegray},
    stringstyle = \color{codepurple},
    basicstyle = \ttfamily\footnotesize,
    breakatwhitespace = false,
    breaklines = true,
    captionpos = b,
    keepspaces = true,
    numbers = left,
    numbersep = 5pt,
    showspaces = false,
    showstringspaces = false,
    showtabs = false,
    tabsize = 2
}

\lstset{
  style = mystyle,
  inputencoding = utf8,
  extendedchars = true,
  literate = {á}{{\'a}}1 {é}{{\'e}}1 {í}{{\'i}}1 {ó}{{\'o}}1 {ú}{{\'u}}1
             {Á}{{\'A}}1 {É}{{\'E}}1 {Í}{{\'I}}1 {Ó}{{\'O}}1 {Ú}{{\'U}}1
             {ñ}{{\~n}}1 {Ñ}{{\~N}}1 {π}{{$\pi$}}1 {≈}{{$\approx$}}1
}

% Hiperreferencias al final para evitar conflictos
\usepackage[colorlinks]{hyperref}
\hypersetup{pdfencoding = auto}


\title{Extracción y perfilado de datos}

\begin{document}

\lhead{Extracción y perfilado de datos}

\title{Práctica 4 - Extracción y perfilado de datos}

\thispagestyle{empty}

\begin{figure}[ht]
	\minipage{0.75\textwidth}
	\includegraphics[width = 4.6 cm]{logo_unam.png}
	\endminipage
	\minipage{0.32\textwidth}
	\includegraphics[height = 4.9cm, width = 4cm]{logo_iimas.png}
	\endminipage 
\end{figure}

\begin{center}
	\vspace{0.5cm}
	\LARGE
	UNIVERSIDAD NACIONAL AUTÓNOMA DE MÉXICO

	\vspace{0.5cm}
	\LARGE
	Instituto de Investigaciones en Matemáticas Aplicadas y Sistemas

	\vspace{1.5cm}
	\Large
	\textbf{Práctica 4 DWH - Extracción y perfilado de datos}

	\vspace{1cm}
	\normalsize
	PRESENTA \\
	\vspace{.3cm}
	\large
	\textbf{Díaz Juárez Ana Sofía \\}
	\textbf{Munive Ramírez Ibrahim \\}
	\textbf{Pérez Aguiar Oropeza Gabriel Emiliano \\}


	\vspace{1cm}
	\normalsize
	PROFESORA \\
	\vspace{.3cm}
	\large
	\textbf{Dra. María del Pilar Angeles}

	\vspace{1cm}
	\normalsize
	ASIGNATURA \\
	\vspace{.3cm}
	\large
	\textbf{Bases de Datos Estructuradas}

	\vspace{1cm}
\end{center}

\newpage


\section*{Contexto}

Una vez creado el repositorio ServiciosFinancieros, es necesario
explotar las funetes OLTP con las que se cuenta. Considerar el 
archivo \texttt{database-sucia-citi.csv} que contiene las 
direcciones de las sucursales de Citi Bank que servirá para cargar 
datos limpios a la tabla Branch.

\begin{center}
    \begin{tabular}{|c|}
        \hline
        \underline{\textbf{branch\_key}} \\
        branch\_name \\
        branch\_addres \\
        branch\_city \\
        branch\_state \\
        branch\_zip \\
        branch\_type \\
        \hline
    \end{tabular}
\end{center}

\vspace{0.5 cm}
\hrule
\vspace{0.5 cm}

\section*{Objetivos}

Extraer los datos relevantes de la fuente dada y realizar perfilado 
de calidad de datos para el sistema de Servicios Financieros. 


\vspace{1.5 cm}
\hrule
\vspace{0.5 cm}

\section*{Instrucciones}

1. A partir del archivo \texttt{database-sucia-citi.csv} extraer 
aquellos datos que son relevantes para ser cargados a la tabla 
Branch del repositorio ServiciosFinancieros y guardar el archivo 
'Branch.csv' con campos delimitados por comas con nombre.

\vspace{0.5 cm}

\begin{table}[H]
    \centering
    \resizebox{\textwidth}{!}
    {
        \rowcolors{2}{rowgray}{white}
        \begin{tabular}{llllllllrrrrrrrrr}
            \toprule
            
            \rowcolor{headerblue}
            {
                \color{headertext}\textbf{Institution}
            } &
            {
                \color{headertext}\textbf{Branch Name}
            } &
            {
                \color{headertext}\textbf{Estab. Date}
            } &
            {
                \color{headertext}\textbf{Acq. Date}
            } &
            {
                \color{headertext}\textbf{Street Address}
            } &
            {
                \color{headertext}\textbf{County}
            } &
            {
                \color{headertext}\textbf{State}
            } &
            {
                \color{headertext}\textbf{Zipcode}
            } &
            {
                \color{headertext}\textbf{Latitude}
            } &
            {
                \color{headertext}\textbf{Longitude}
            } &
            {
                \color{headertext}\textbf{2010 Dep}
            } &
            {
                \color{headertext}\textbf{2011 Dep}
            } &
            {
                \color{headertext}\textbf{2012 Dept}
            } &
            {
                \color{headertext}\textbf{2013 Dept}
            } &
            {
                \color{headertext}\textbf{2014 Dept}
            } &
            {
                \color{headertext}\textbf{2015 Dept}
            } &
            {
                \color{headertext}\textbf{2016 Depo}
            } \\
            
            \midrule
            Citi Bank & O boston branch 4 de enero de 2003 & & 08/16/2008 & 50 ROWES WHARF, FLOOR 4 & Suffolk & MA & 2110 & 42.35397 & -71.0498 & 0 & 0 & 0 & 0 & 0 & 0 & 0 \\
            Citi Bank & O milford (ford s/09/16/1957 & & 07/14/1996 & 370 BOSTON POST ROAD & New Haven & CT & 6460 & 41.22554 & -73.0712 & 82623 & 83772 & 86939 & 86705 & 87268 & 89086 & 99210 \\
            Citi Bank & O milford bankin 7 de enero de 2010 & & & 1651 BOSTON POST ROAD & & CT & 6460 & 41.24732 & -73.0252 & & 5102 & 11047 & 18696 & 24017 & 28134 & 33414 \\
            Citi Bank & O monroe ct brar 8 de agosto de 1979 & & 07/14/1996 & 456 MONROE TURNPIKE & Fairfield & CT & 6468 & 41.31428 & -73.2184 & 63203 & 66795 & 68238 & 76666 & 79725 & 82939 & 84911 \\
            Citi Bank & O newtown bran 12/28/1974 & & 07/14/1996 & 30 CHURCH HILL ROAD & Fairfield & CT & 6470 & 41.41458 & -73.3016 & 51729 & 58509 & 65722 & 76224 & 84761 & 87607 & 100543 \\
            Citi Bank & O boston post ro 12 de agosto de 2009 & & & 262 BOSTON POST ROAD & New Haven & CT & 6477 & 41.26741 & -73.0007 & 11091 & 22112 & 35691 & 47681 & 70920 & 76043 & 82218 \\
            Citi Bank & O shelton branch 7 de octubre de 1982 & & 07/14/1996 & 675 BRIDGEPORT AVENUE & Fairfield & CT & 6484 & 41.27746 & -73.1215 & 67941 & 62727 & 65088 & 70702 & 85390 & 82766 & 93207 \\
            Citi Bank & O southbury grec 11 de julio de 2011 & & & 775 MAIN STREET SOUTH & New Haven & CT & 6488 & 41.46931 & -73.2346 & & 6638 & 16492 & 27050 & 35887 & 45501 & \\
            \bottomrule
        \end{tabular}
    }
    \caption{Tabla fea de la práctica.}
    \label{table:1}
\end{table}

\vspace{0.5 cm}

Primero pasaremos el archivo \texttt{database-sucia-citi.csv} 
a DataFrame, esto para poder visualizar las columnas sus datos, 
y en base a estos (porque los encabezados no nos dicen mucho), 
pasar la información al formato de la tabla Branch.

\vspace{1.5 cm}

Recordemos que la tabla \texttt{Branch} tiene la siguiente estructura:

\begin{table}[H]
    \centering
    \resizebox{\textwidth}{!}
    {
        \rowcolors{2}{rowgray}{white}
        \begin{tabular}{cllccc}
            \toprule
            
            \rowcolor{headerblue}
            {
                \color{headertext}\textbf{branch\_key}
            } &
            {
                \color{headertext}\textbf{branch\_name}
            } &
            {
                \color{headertext}\textbf{branch\_address}
            } &
            {
                \color{headertext}\textbf{branch\_city}
            } &
            {
                \color{headertext}\textbf{branch\_state}
            } &
            {
                \color{headertext}\textbf{branch\_zip}
            } \\
            
            \midrule
            1 & Big Apple  & 822 Financial Way  & New York     & NY & 46842 \\
            2 & Philly     & 15 Financial Way   & Philadelphia & PA & 36769 \\
            3 & Windy City & 923 Financial Way  & Chicago      & IL & 43389 \\
            4 & Tinseltown & 945 Financial Way  & Los Angeles  & CA & 80626 \\
            \bottomrule
        \end{tabular}
    }
    \caption{Estructura de la tabla \texttt{Branch}.}
    \label{table:2}
\end{table}

Nota: en esta tabla ignoramos el \texttt{branch\_type} porque no encontramos un 
equivalente en el archivo csv (y porque no viene en la tabla de la práctica), 
pero lo agregaremos en una columna sin contenido.

\vspace{0.5 cm}

Usamos \texttt{Python} para hacer la extracción de los datos del archivo csv, 
pues es lo que mejor sabemos usar. 

Los pasos a seguir para el \texttt{script1} fueron:

\begin{enumerate}
    \item Importar las bibliotecas necesarias 
    \item Leer el archivo \texttt{database-sucia-citi.csv} 
    \item Convertir el archivo original a DataFrame
    \item Renombrar las columnas que tienen la información solicitada.
    \item Hacer una copia del DataFrame, para no modificar el original.
    \item Crear las columas que nos hagan falta y no tengan un equivalente.
    \item Llenar las columnas creadas con la información que sea necesaria. 
    \item Ordenar las columnas conforme a los requerimientos.
    \item Pasar el DataFrame a \texttt{csv} y guardarlo.
    \item Comprobar los cambios realizados.
\end{enumerate} 

\vspace{0.5 cm}

\lstinputlisting[
    language = Python,
    caption = {Extracción de datos.},
    label = {lst:script1}
]{D:/IIMAS/Base de datos/P4 - DWH - Extraccion y perfilado/script1.py}

\vspace{0.5 cm}

\subsection*{Resultados}

\begin{figure}[H]
    \centering
    \includegraphics[width = 1\textwidth]{TablaBranch.png}
    \caption{Tabla \texttt{Branch.csv} (vista desde Jupyter Notebook).}
    \label{fig:TB}
\end{figure}

\newpage

\newpage

2. Obtener el perfil de los datos almacenados en \texttt{Branch.csv} 
siguiendo el procedimiento:
\begin{enumerate}
    \item Explorar los datos contenidos en \texttt{Branch.csv}
    
    \item Idenfificar y anotar las características que tiene 
        cada campo, por ejemplo: si tiene valores faltances, 
        falta de homogeneidad en el formato de los textos, etc.
    
    \item Generar las siguientes estadísticas utilizando el 
        software de su preferencia:
        \begin{itemize}
            \item[$\mathcal{A}_1$.] Número de registros del archivo
            \item[$\mathcal{A}_2$.] Número de valores faltantes
            \item[$\mathcal{A}_3$.] Número de registros duplicados
            \item[$\mathcal{A}_4$.] Número de códigos postales de 
                Disparos Unidos de América válidos     
        \end{itemize}
    
    \item Identificar otros indicadores estadísticos que puedan 
        servir para el análisis OLAP
    
    \item Identificar si existe información que se requiera en la 
        tabla Branch que no la proporcione \texttt{database-sucia-citi.csv}
        y explciar cómo completar dicha información. 
\end{enumerate}

\vspace{0.3 cm}

\subsection*{Paso 1}

Para este primer paso, leemos el archivo \texttt{Branch.csv} en un 
try, para evitar que el programa truene por un archivo sin encontrar. 
Y mostramos el encabezado (las primeras filas) del conjunto de datos.

\vspace{0.5 cm}

\lstinputlisting[
    language = Python,
    firstline = 4,
    lastline = 13,
    caption = {Exploración de los datos de \texttt{Branch.csv}.}
]{D:/IIMAS/Base de datos/P4 - DWH - Extraccion y perfilado/script2.py}

\vspace{1.5 cm}

\subsection*{Paso 2}

Aquí se detectan valores faltantes y formatos inconsistente.

\vspace{0.5 cm}

\lstinputlisting[
    language = Python,
    firstline = 15,
    lastline = 21,
    caption = {Características de campo.}
]{D:/IIMAS/Base de datos/P4 - DWH - Extraccion y perfilado/script2.py}

\vspace{0.5 cm}

\begin{lstlisting}[language = Python, caption = {Información de los datos.}]
Primeras filas del archivo Branch.csv:

   branch_key                    branch_name           branch_address  \
0           1                  boston branch  50 ROWES WHARF, FLOOR 4   
1           2   milford (ford street) branch     370 BOSTON POST ROAD   
2           3  milford banking center branch    1651 BOSTON POST ROAD   
3           4               monroe ct branch      456 MONROE TURNPIKE   
4           5                 newtown branch      30 CHURCH HILL ROAD   

  branch_city branch_state  branch_zip  branch_type  
0     Suffolk           MA      2110.0          NaN  
1   New Haven           CT      6460.0          NaN  
2         NaN           CT      6460.0          NaN  
3   Fairfield           CT      6468.0          NaN  
4   Fairfield           CT      6470.0          NaN  
\end{lstlisting}

\vspace{0.5 cm}

Aquí se puede identificar que a \texttt{branch\_zip} le falta 
homogeneidad de formato, pues son flotantes, lo cual no nos 
sirve para un código postal, así que podríamos pasarlos a un 
dato cadena. Por otro lado, \texttt{branch\_type} está vació, 
pues fue una columna creada que no tenía algún equivalente 
en el archivo original. 

\vspace{0.5 cm}

\begin{lstlisting}[language = Python, caption = {Información de la tabla.}]
Información general del DataFrame:

<class 'pandas.core.frame.DataFrame'>
RangeIndex: 5445 entries, 0 to 5444
Data columns (total 7 columns):
 #   Column          Non-Null Count  Dtype  
---  ------          --------------  -----  
 0   branch_key      5445 non-null   int64  
 1   branch_name     5445 non-null   object 
 2   branch_address  5386 non-null   object 
 3   branch_city     5342 non-null   object 
 4   branch_state    5388 non-null   object 
 5   branch_zip      5431 non-null   float64
 6   branch_type     0 non-null      float64
dtypes: float64(2), int64(1), object(4)
memory usage: 297.9+ KB
None
\end{lstlisting}

\vspace{0.5 cm}

En esta salida se visualiza el total de columnas y de registros 
(en la columna \textbf{branch\_key}) y que tenemos columnas con 
datos faltantes. Y hay una falta de homogeneidad de formato en los nombres 
y direcciones. Es decir, todo esto indica que los datos necesitan una 
limpieza adicional antes de cargarse a la base de datos final. 

\vspace{0.5 cm}

\subsection*{Paso 3}

\lstinputlisting[
    language = Python,
    firstline = 25,
    lastline = 50,
    caption = {Estadísticas}
]{D:/IIMAS/Base de datos/P4 - DWH - Extraccion y perfilado/script2.py}

\vspace{0.5 cm}

En esta parte vemos los registro totales, cada uno correspondiente con una 
sucursal, los datos faltantes que, nos indican problemas de completitud. Y 
la completitud nos da una referencia de cúan completa está una columna, 
marca $0$ en \texttt{branch\_type}, que infica que no hay registros y muestra 
un $0.981$ en la columna \texttt{branch\_city}, que no está de todo completa. 

También da el número de registros duplicados; lo cual indica cuántas filas 
son exactamente iguales en todas las columnas.
Si el resultado es 0, no hay duplicados y los datos son únicos.
En caso de que existan, deben eliminarse antes de cargar la tabla

Se contaron lo códigos postales de cinco dígitos. Para esto, se cambian en 
la columna \texttt{branch\_zip} (si es que existe) los valores 
\textcolor{purple}{\texttt{NaN}} por una cadena vacía $''$ para evitar errores. 
Luego se convierte toda la columna a texto para solucionar la falta de homogeneidad 
anteriormente vista. Después eliminamos los los caracteres $.0$ al final de la cadena 
y con una expresión regular, verificamos que nuestra cadena tenga únicamente cinco 
caracteres (le indicamos el inicio y el fin de la cadena con $\land$ y \$ respectivamente).
Finalmente, devolvemos un \texttt{True} o \texttt{False} si se cumple con la expresión 
regular y contamos todos los \texttt{True}. 

\vspace{0.5 cm}

\begin{lstlisting}[language = Python, caption = {Estadísticas}]
Total de registros = 5445

Valores faltantes por campo:
branch_key           0
branch_name          0
branch_address      59
branch_city        103
branch_state        57
branch_zip          14
branch_type       5445

Completitud campo:
branch_key        1.000
branch_name       1.000
branch_address    0.989
branch_city       0.981
branch_state      0.990
branch_zip        0.997
branch_type       0.000

Registros duplicados = 0

Códigos postales válidos = 5129
\end{lstlisting}

\vspace{2.5 cm}

\subsection*{Paso 4}

Otro indicador puede ser el total de valores únicos por columa.

\vspace{0.5 cm}

\lstinputlisting[
    language = Python,
    firstline = 52,
    lastline = 54,
    caption = {Características de campo.},
]{D:/IIMAS/Base de datos/P4 - DWH - Extraccion y perfilado/script2.py}

\vspace{0.5 cm}

\begin{lstlisting}[language = Python, caption = {Valores únicos por campo.}]
Valores únicos por campo:
branch_key        5445
branch_name       5341
branch_address    5320
branch_city        378
branch_state        30
branch_zip        3544
branch_type          0
\end{lstlisting}

\vspace{0.5 cm}

\subsection*{Paso 5}

Si la tabla Branch requiere columnas que no existen en la tabla original, 
sim limpiar, como \texttt{branch\_type} u otras, esa información deberá 
completarse manualmente o con otra fuente.

\QED

\end{document}
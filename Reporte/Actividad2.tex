Obtener el perfil de los datos almacenados en \texttt{Branch.csv} 
siguiendo el procedimiento

\begin{enumerate}
    \item Explorar los datos contenidos en \texttt{Branch.csv}
    
    \item Idenfificar y anotar las características que tiene 
        cada campo, por ejemplo: si tiene valores faltances, 
        falta de homogeneidad en el formato de los textos, etc.
    
    \item Generar las siguientes estadísticas utilizando el 
        software de su preferencia:
        \begin{itemize}
            \item[$\mathcal{A}1$.] Número de registros del archivo
            \item[$\mathcal{A}2$.] Número de valores faltantes
            \item[$\mathcal{A}3$.] Número de registros duplicados
            \item[$\mathcal{A}4$.] Número de códigos postales de 
                Disparos Unidos de América válidos     
        \end{itemize}
    
    \item Identificar otros indicadores estadísticos que puedan 
        servir para el análisis OLAP
    
    \item Identificar si existe información que se requiera en la 
        tabla Branch que no la proporcione \texttt{database-sucia-citi.csv}
        y explciar cómo completar dicha información. 
\end{enumerate}
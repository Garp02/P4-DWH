2. Obtener el perfil de los datos almacenados en \texttt{Branch.csv} 
siguiendo el procedimiento:
\begin{enumerate}
    \item Explorar los datos contenidos en \texttt{Branch.csv}
    
    \item Idenfificar y anotar las características que tiene 
        cada campo, por ejemplo: si tiene valores faltances, 
        falta de homogeneidad en el formato de los textos, etc.
    
    \item Generar las siguientes estadísticas utilizando el 
        software de su preferencia:
        \begin{itemize}
            \item[$\mathcal{A}_1$.] Número de registros del archivo
            \item[$\mathcal{A}_2$.] Número de valores faltantes
            \item[$\mathcal{A}_3$.] Número de registros duplicados
            \item[$\mathcal{A}_4$.] Número de códigos postales de 
                Disparos Unidos de América válidos     
        \end{itemize}
    
    \item Identificar otros indicadores estadísticos que puedan 
        servir para el análisis OLAP
    
    \item Identificar si existe información que se requiera en la 
        tabla Branch que no la proporcione \texttt{database-sucia-citi.csv}
        y explciar cómo completar dicha información. 
\end{enumerate}

\vspace{0.3 cm}

\subsection*{Paso 1}

Para este primer paso, leemos el archivo \texttt{Branch.csv} en un 
try, para evitar que el programa truene por un archivo sin encontrar. 
Y mostramos el encabezado (las primeras filas) del conjunto de datos.

\vspace{0.5 cm}

\lstinputlisting[
    language = Python,
    firstline = 4,
    lastline = 13,
    caption = {Exploración de los datos de \texttt{Branch.csv}.}
]{D:/IIMAS/Base de datos/P4 - DWH - Extraccion y perfilado/script2.py}

\vspace{1.5 cm}

\subsection*{Paso 2}

Aquí se detectan valores faltantes y formatos inconsistente.

\vspace{0.5 cm}

\lstinputlisting[
    language = Python,
    firstline = 15,
    lastline = 21,
    caption = {Características de campo.}
]{D:/IIMAS/Base de datos/P4 - DWH - Extraccion y perfilado/script2.py}

\vspace{0.5 cm}

\begin{lstlisting}[language = Python, caption = {Información de los datos.}]
Primeras filas del archivo Branch.csv:

   branch_key                    branch_name           branch_address  \
0           1                  boston branch  50 ROWES WHARF, FLOOR 4   
1           2   milford (ford street) branch     370 BOSTON POST ROAD   
2           3  milford banking center branch    1651 BOSTON POST ROAD   
3           4               monroe ct branch      456 MONROE TURNPIKE   
4           5                 newtown branch      30 CHURCH HILL ROAD   

  branch_city branch_state  branch_zip  branch_type  
0     Suffolk           MA      2110.0          NaN  
1   New Haven           CT      6460.0          NaN  
2         NaN           CT      6460.0          NaN  
3   Fairfield           CT      6468.0          NaN  
4   Fairfield           CT      6470.0          NaN  
\end{lstlisting}

\vspace{0.5 cm}

Aquí se puede identificar que a \texttt{branch\_zip} le falta 
homogeneidad de formato, pues son flotantes, lo cual no nos 
sirve para un código postal, así que podríamos pasarlos a un 
dato cadena. Por otro lado, \texttt{branch\_type} está vació, 
pues fue una columna creada que no tenía algún equivalente 
en el archivo original. 

\vspace{0.5 cm}

\begin{lstlisting}[language = Python, caption = {Información de la tabla.}]
Información general del DataFrame:

<class 'pandas.core.frame.DataFrame'>
RangeIndex: 5445 entries, 0 to 5444
Data columns (total 7 columns):
 #   Column          Non-Null Count  Dtype  
---  ------          --------------  -----  
 0   branch_key      5445 non-null   int64  
 1   branch_name     5445 non-null   object 
 2   branch_address  5386 non-null   object 
 3   branch_city     5342 non-null   object 
 4   branch_state    5388 non-null   object 
 5   branch_zip      5431 non-null   float64
 6   branch_type     0 non-null      float64
dtypes: float64(2), int64(1), object(4)
memory usage: 297.9+ KB
None
\end{lstlisting}

\vspace{0.5 cm}

En esta salida se visualiza el total de columnas y de registros 
(en la columna \textbf{branch\_key}) y que tenemos columnas con 
datos faltantes. Y hay una falta de homogeneidad de formato en los nombres 
y direcciones. Es decir, todo esto indica que los datos necesitan una 
limpieza adicional antes de cargarse a la base de datos final. 

\vspace{0.5 cm}

\subsection*{Paso 3}

\lstinputlisting[
    language = Python,
    firstline = 25,
    lastline = 50,
    caption = {Estadísticas}
]{D:/IIMAS/Base de datos/P4 - DWH - Extraccion y perfilado/script2.py}

\vspace{0.5 cm}

En esta parte vemos los registro totales, cada uno correspondiente con una 
sucursal, los datos faltantes que, nos indican problemas de completitud. Y 
la completitud nos da una referencia de cúan completa está una columna, 
marca $0$ en \texttt{branch\_type}, que infica que no hay registros y muestra 
un $0.981$ en la columna \texttt{branch\_city}, que no está de todo completa. 

También da el número de registros duplicados; lo cual indica cuántas filas 
son exactamente iguales en todas las columnas.
Si el resultado es 0, no hay duplicados y los datos son únicos.
En caso de que existan, deben eliminarse antes de cargar la tabla

Se contaron lo códigos postales de cinco dígitos. Para esto, se cambian en 
la columna \texttt{branch\_zip} (si es que existe) los valores 
\textcolor{purple}{\texttt{NaN}} por una cadena vacía $''$ para evitar errores. 
Luego se convierte toda la columna a texto para solucionar la falta de homogeneidad 
anteriormente vista. Después eliminamos los los caracteres $.0$ al final de la cadena 
y con una expresión regular, verificamos que nuestra cadena tenga únicamente cinco 
caracteres (le indicamos el inicio y el fin de la cadena con $\land$ y \$ respectivamente).
Finalmente, devolvemos un \texttt{True} o \texttt{False} si se cumple con la expresión 
regular y contamos todos los \texttt{True}. 

\vspace{0.5 cm}

\begin{lstlisting}[language = Python, caption = {Estadísticas}]
Total de registros = 5445

Valores faltantes por campo:
branch_key           0
branch_name          0
branch_address      59
branch_city        103
branch_state        57
branch_zip          14
branch_type       5445

Completitud campo:
branch_key        1.000
branch_name       1.000
branch_address    0.989
branch_city       0.981
branch_state      0.990
branch_zip        0.997
branch_type       0.000

Registros duplicados = 0

Códigos postales válidos = 5129
\end{lstlisting}

\vspace{2.5 cm}

\subsection*{Paso 4}

Otro indicador puede ser el total de valores únicos por columa.

\vspace{0.5 cm}

\lstinputlisting[
    language = Python,
    firstline = 52,
    lastline = 54,
    caption = {Características de campo.},
]{D:/IIMAS/Base de datos/P4 - DWH - Extraccion y perfilado/script2.py}

\vspace{0.5 cm}

\begin{lstlisting}[language = Python, caption = {Valores únicos por campo.}]
Valores únicos por campo:
branch_key        5445
branch_name       5341
branch_address    5320
branch_city        378
branch_state        30
branch_zip        3544
branch_type          0
\end{lstlisting}

\vspace{0.5 cm}

\subsection*{Paso 5}

Si la tabla Branch requiere columnas que no existen en la tabla original, 
sim limpiar, como \texttt{branch\_type} u otras, esa información deberá 
completarse manualmente o con otra fuente.